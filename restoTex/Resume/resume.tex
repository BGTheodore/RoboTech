\section*{Résumé}
\paragraph{}
Le client branché entre par une porte spéciale. Son 
nom est saisi dans un système de gestion. Une table 
lui est alors assignée, ainsi qu'un nuuméro unique.
À noter que plusieurs clients peuvent partager une 
table. Ils ne seront simplement pas tous enregistrés 
individuellement, mais sous un seul identifiant. \par 
Le numéro une fois communiquée au client, un point 
lumineux indique sa place sur un plan (affiché sur un 
mur situé en face de l'entrée). Dès que le client 
s'est installé, le robot vient à sa table pour prendre
sa commande. Une fois la commande reçue, le robot 
transmet à un barman en charge de préparer le plateu
qui est ensuite déposé dans un receptacle où le robot 
le prend pour l'apporter au client, repéré par son 
numéro. \par 
Chaque table dispose d'un petit panneau pour passer
des instructions. Ce panneau comporte trois boutons:
\par 
\begin{itemize}
    \item Service: le client souhaite placer une nouvelle commande
    \item Facture: le client demande sa facture afin de payer ses consommations
    \item Fin: le client se lève et libère la place qu'il occupait.
    Après avoir payé sa facture, si la place n'est pas 
    libérée dans un délai de dix (10) minutes, une alerte
    est envoyée au gestionnaire de la salle avec le 
    numéro et le nom du client.
\end{itemize}
\par
Le robot a la charge d'apporter au client sa facture. 
Après validation de cette dernière, le client quitte sa
place et laisse le restaurant.